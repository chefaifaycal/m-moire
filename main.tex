\documentclass[french]{report}
\usepackage[utf8]{inputenc}
\usepackage[T1]{fontenc}
\usepackage{fourier}
\usepackage{babel}
\usepackage{graphicx}
\usepackage[x11names, table]{xcolor}
\usepackage{float}
\usepackage{csquotes}
\usepackage{array}
\usepackage{multirow}
\usepackage{biblatex}
\addbibresource{Biblio.bib}

\newcommand\rmq[1]{\textcolor{red}{\tt #1}}

\title{Mémoire Master}
\author{faycalchefai1 }
\date{September 2021}

\begin{document}
%Page de garde


\begin{titlepage}
\begin{center}
    \large{République Algérienne Démocratique et Populaire}\\
    \large{Ministère de l'Enseignement Supérieur et de la Recherche Scientifique}\\
    \large{Université A. Mira de Bejaia}\\
    \large{Faculté des Sciences Exactes}\\
    \large{Département d'Informatique}\\

    \begin{figure}[h!]
    \centering
    \includegraphics[width=6cm]{images/logo1.jpg}
    \end{figure}
	
    \LARGE{\textbf{Mémoire de Fin de Cycle}} \\[2ex]
     En vue de l'obtention du diplôme de Master en Informatique \\[1ex]
    
    \textit{\textbf{Thème}}\\[1ex]
	\rule{11,5cm}{1pt}
    {\LARGE{\textbf{{\\Conception et réalisation d'une application Web d'offres
    de services professionnels \\}}}}
	\rule{11,5cm}{1pt}\\
	\vspace{0.2cm}
\end{center}

\begin{center}
    \large Réalisé par

    \paragraph{}

    \begin{tabular}{l l l l}	
        M. CHEFAI Fayçal & & & M. BORDJIHANE Sami \\
    \end{tabular}

    \large Devant le jury composé de
\end{center}
\paragraph{}

\begin{table}[h!]
    \centering
    \begin{tabular}{l l l }
     & Promotion 2020 - 2021 &	\\
    \end{tabular}    
\end{table}
\end{titlepage}

%Fin de page de garde

\tableofcontents
\listoffigures
\listoftables


%Début introduction
\chapter*{Introduction Générale}

De nos jours, Internet a pris un grand élan dans notre vie. Avant il n'était
destinée qu'aux experts et aux grandes entreprises. Aujourd'hui, il est
devenue indispensable et offre des services dans tous les domaines. Internet
permet aux entreprises ou même aux boutiques d'offrir leurs services afin de
mieux répondre aux besoins de ses clients. Dans notre projet, nous nous
intéressons à la gestion de ces services.  En effet, dans notre vie quotidienne,
lors de la recherche d'une personne professionnelle (ou compétente) dans un
domaine   nous rencontrons beaucoup de difficultés, par exemple : les
informations, les contacts. Ce qui  peut engendrer une perte de temps et
d'argent. Pour pallier ces problèmes, nous avons créé  et développé une
application web dynamique de gestion d'offres des services. A travers cette
application, on regroupe plein de gens compétents qui peuvent partager leurs
travaux et leurs contacts. Ainsi, toute personne pourra consulter ou bien
acheter des services. 

Notre choix de développer une application web, se justifie par le fait qu'elle
est un moyen très utile et efficace pour exposer et offrir des services aux
internautes via internet grâce à l'accessibilité qu'elle offre aux utilisateurs
ou bien les clients indépendamment de leurs emplacements physiques, ce qui nous
permettra de rapprocher les fournisseur et les clients.  Dans le présent rapport,
nous allons présenter les différentes étapes que nous avons suivies lors du
développement de notre application web. Notamment, la démarche suivie, les
diagrammes conçus, ainsi que les différents langages et outils utilisés. Enfin, 
nous clôturons ce travail avec une conclusion générale et quelques
perspectives qui permettront d'améliorer notre solution.

\rmq{Développez davantage cette introduction générale. Illustrez, donnez des
exemples, \ldots Vous avez l'obligation d'intéresser le lecteur et de le
convaincre de l'utilité de votre travail.}

%Fin introduction


%Début du chapitre 1
\chapter{Étude Préalable}
\section{Introduction}
De nos jours les désirs ne sont plus comme avant (\rmq{reformulez}), aujourd'hui
nous nous focalisons sur la réalisation des besoins efficacement en un temps
minimal. C'est pour cela que nous optons pour les applications qui offrent des
services en ligne sans avoir à se déplacer ni à perdre beaucoup de temps et
d'argent.  Dans le cadre de notre travail, nous nous focalisons sur la gestion
d'offre de services en ligne, notamment sur la publication des services.
L'objectif principal est de faciliter la tâche à l'internaute concernant ses
informations, ses contacts, ses achats,  etc.

\section{Identification de la problématique}

Le problème que rencontre beaucoup de gens dans leurs vie quotidienne, est la
recherche de professionnels compétents. En outre, ils ont essentiellement besoin
d'informations telles que:

\begin{itemize}
    \item Contacts de professionnels.
    \item Tarifs usités.
    \item Disponibilité.
    \item Travaux antérieurs.
    \item Fiabilité. 
\end{itemize}

\section{Présentation de la solution proposée}

Pour remédier à la problématique énoncée ci-dessus, nous proposons une solution
informatique sous forme d'application Web fournissant les fonctionnalités
suivantes:

\begin{itemize}
    \item Recherche filtrée de services (selon le tarifs, le domaine, le lieu,
        \ldos).
    \item Accès aux informations liées au service (tarifs, durée de réalisation,
        \ldots).
    \item Possibilité de contacter le fournisseur du service.
    \item Possibilité de commander un service.
    \item Possibilité de suivi de ses commandes.
    \item Mise en place d'un système de notation des services.
    \item Ajout de nouveaux services.
    \item Gestion des services (édition, suppression).
\end{itemize}

\section{Besoins fonctionnels}

\begin{itemize}
    \item Le système permettra aux clients et aux fournisseurs de communiquer.
    \item Le système permettra aux clients de chercher, consulter, et commander
        un service.
    \item Le système permettra aux clients d'ajouter des avis.
    \item Le système permettra aux fournisseurs d'ajouter, modifier, supprimer
        un service.
    \item Le système permettra aux fournisseurs de gérer leurs services.
        (\rmq{comment? C'est pas le point précédent ça?})
    \item Le système permettra d'afficher quelques statistiques pour les
        fournisseurs.
\end{itemize}

\section{Besoins non fonctionnels}

\begin{itemize}
    \item Vérifier et valider l'identité d'une personne ou
        l'identification de toute autre entité pour contrôler l'accès à un
        réseau.
    \item Choisir des solutions adéquates pour protéger l'utilisateur.
    \item La capacité du site à aider à la réalisation de l'activité
        d'utilisateur pour laquelle il a été conçu.
    \item La capacité du site à être opérationnel à tout instant.
\end{itemize}

\section{Conclusion}

Dans se chapitre nous avons déterminé la problématique rencontrée par la plus
part des gens lors de la recherche d'un professionnel, puis nous avons suggéré
une solution informatique sous forme d'application web.(\rmq{Pas suffisant comme
conclusion})

%Fin du chapitre 1

%Début du chapitre 2
\chapter{Analyse et Conception}

\section{Introduction}

L'analyse et la conception est un procédé essentiel du processus de
développement qui a pour objectif de formaliser les étapes préliminaires du
développement d'un système afin de satisfaire les besoins du client.

Dans ce chapitre, nous mettrons en évidence le coté conceptuel de notre
application. Nous utiliserons une démarche généralisée basé sur UP qui utilise
le langage UML. Les diagrammes UML utilisés sont les suivant: Diagramme de cas
d'utilisation, diagrammes de séquence, et les diagrammes de classes\rmq{Pas la
peine de le dire(, qui sont modélisés l'aide de l'outil Draw.io.)}

\section{Le processus unifié}

Le processus unifié est un processus de développement logiciel orientés objets,
centré sur l'architecture, piloté par des cas d'utilisation, et orienté vers la
diminution des risques. C'est une méthode générique et itérative \cite{UP}
\rmq{Vous pouvez en dire plus?}

\begin{figure}[h] 
    \center 
    \includegraphics[width=\linewidth]{images/Screenshot(1).png} 
    \caption{UP est itératif}
\end{figure}

\section{Langage de modélisation UML}

UML est né de la fusion des 3 principales méthodes d'analyse et de conception
parues pendant les années 90. Ces méthodes sont OMT (Object Modeling Technique),
BOOCH, et OOSE.\rmq{(référence)}

James Rumbaugh (créateur de OMT) et GradyBooch (créateur de BOOCH) se sont
retrouvés au sein de la société \emph{Rational Software} et ont été ensuite
rejoints par lavr Jacobson (créateur de OOSE) en se donnant comme objectif de
fusionner leur méthodes et créer UML (UnifiedMethodeLanguage).

UML est donc une norme du langage de modélisation objet qui a été publiée, dans
sa première version, en novembre 1997 par l'OMG (Object Managment Group),
instance de normalisation internationale du domaine de l'objet.

\subsection{Définition}

L'acronyme UML signifie \emph{``Langage de modélisation unifié"}. C'est un
langage visuel constitué d'un ensemble de schémas appelés \emph{diagrammes}, qui
modélisent les besoins du logiciel. Chaque diagramme donne une vision différente
du projet et représente le logiciel à développer: son fonctionnement, sa mise en
route, les actions susceptibles d'être effectuées par le logiciel, etc.
\cite{UML2}

\begin{figure}[!h] 
    \center 
    \includegraphics[width=1\linewidth]{images/uml-logo-large.jpg} 
    \caption{UML Logo}
\end{figure}

\subsection{Les diagrammes}

\begin{description}
    \item[Diagramme de classes :]
        Les diagrammes de classes expriment de manière générale la structure
        statique d'un système, en termes de classes et de relations entre ces
        classes. Outre les classes, ils représentent un ensemble d'interfaces
        et de paquetages, ainsi que leurs relations.\cite{UML3}

    \item[Diagramme de cas d'utilisation:]
        Les diagrammes de cas d'utilisation représentent un ensemble \rmq{de cas
        d'utilisation}, d'acteurs, et leurs relations. Ils représentent la vue
        statique des cas d'utilisation d'un système et sont particulièrement
        importants dans l'organisation et la modélisation du comportement d'un
        système. \cite{UML3}

    \item[Diagramme de séquence :]
        Un diagramme de séquence met en évidence le classement des messages par
        ordre chronologique. On forme un diagramme de séquence en plaçant
        d'abord les objets qui participent à l'interaction en haut du diagramme.
        Le long de l'axe des abscisses. En général, on place l'objet qui débute
        l'interaction à gauche, puis on continue en progressant vers la droite,
        les objets les plus subordonnés étant tout à fait à droite. On place
        ensuite les messages envoyés et reçus par ces objets le long de l'axe
        des ordonnées, par ordre chronologique, du haut vers le bas. Cela donne
        au lecteur une indication visuelle claire du flot de contrôle dans le
        temps. \cite{UML3}
\end{description}

\subsection{Identification des acteurs et leurs rôles }

Dans notre système, on distingue les acteurs suivants :

\begin{table}[h!]
    \centering
    \begin{tabular}{ |m{4cm}|m{4cm}|m{4cm}| }
    \hline
    \textbf{Acteurs} & \textbf{Codification} & \textbf{Rôle} \\
    \hline
    Administrateur & Admin & Gérer le système en effectuant des opérations
        d'ajout, suppression et modification sur la base de données.\\
    \hline
    Client & Client & Inscription, connexion,  consultation de service,
        Contacter des fournisseur, laisser des avis, commander des services,
        gérer ses commandes.\\
    \hline
    Fournisseur & Fournisseur & Inscription, connexion, gestion de services,
        gestion de la messagerie.\\
    \hline
    \end{tabular}
    \caption{Identification des acteurs et leurs rôles}
    \label{Tableaux:1}
\end{table}

\subsection{Identification des cas d'utilisations}
\begin{table}[H]
    \centering
    \begin{tabular}{|m{5cm}|m{5cm}|}
        \hline
        \textbf{Cas d'utilisation} & \textbf{Acteurs} \\
        \hline
        Inscription & \multirow{5}{5cm}{Client, Fournisseur} \\
        Connexion & \\
        Gestion des commandes & \\
        Gestion des avis & \\
        Gestion des messages & \\
        \hline
        Consultation de services & \multirow{2}{5cm}{Client} \\
        Recherche de services & \\
        \hline
        Ajout de services & \multirow{3}{5cm}{Fournisseur} \\
        Suppression de services & \\
        Modification de services & \\
        \hline
        Connexion & \multirow{7}{5cm}{Administrateur} \\
        Gestion des clients & \\
        Gestion des fournisseurs & \\
        Gestion des messages & \\
        Gestion des commandes & \\
        Gestion des avis & \\
        Gestion des services & \\
        \hline
    \end{tabular}
    \caption{Identification des cas d'utilisation}
    \label{tab:my_label}
\end{table}

\subsection{Diagrammes de cas d'utilisation}

\begin{itemize}
    \item \underline{Générale:}
        \begin{figure}[H]
            \centering
            \includegraphics[width=1.1\textwidth]{images/use case.png}
            \caption{Diagramme de cas d'utilisation générale}
            \label{fig:general_use_case}
        \end{figure}
        
    \item \underline{Client :} 
        \begin{figure}[H]
            \centering
            \includegraphics[width=1.1\textwidth]{images/client use case diag.png}
            \caption{Diagramme de cas d'utilisation de l'acteur Client}
            \label{fig:client_use_case}
        \end{figure}
    \item \underline{Fournisseur :}
        \begin{figure}[H]
            \centering
            \includegraphics[width=1.1\textwidth]{images/fournisseur use case diag.png}
            \caption{Diagramme de cas d'utilisation de l'acteur Fournisseur}
            \label{fig:fournisseur_use_case}
        \end{figure}
    \item \underline{Administrateur :}
        \begin{figure}[H]
            \centering
            \includegraphics[width=1.1\textwidth]{images/admin use case diag.png}
            \caption{Diagramme de cas d'utilisation de l'acteur Administrateur}
            \label{fig:admin_use_case}
        \end{figure}
\end{itemize}


\subsection{Modélisation des diagrammes de séquence}

Les diagrammes de séquences sont la représentation graphique des interactions
entre les acteurs et le système selon un ordre chronologique dans la formulation
UML. Ces communications entre les classes ‘objets' sont reconnues comme des
messages. Le diagramme des séquences énumère des objets horizontalement, et le
temps verticalement. Il modélise l'exécution des messages en fonction du temps.
\cite{UML3}


\begin{itemize}
\item \textbf{Se connecter} 
\end{itemize}

Le diagramme de séquence suivant illustre les interactions nécessaires pour la
connexion d'un utilisateur. Après l'inscription utilisateur peut se connecter à
l'application, pour cela il n'a qu'à cliquer sur le bouton de connexion en
saisissant son identifiant et son mot de passe, le système vérifie ces champs
auprès de la base de données, si ses informations sont correctes alors il aura
accès à l'application pour gérer les services (comme par exemple ajouter un
service) sinon un message d'erreur sera affiché et le système lui redemande à
nouveau de saisir ses coordonnées.

\begin{figure}[H]
    \centering
    \includegraphics[width=1\textwidth]{images/sequence diag authentification.jpg}
    \caption{Diagramme de séquence du cas d'utilisation "Connexion"}
    \label{fig:my_label}
\end{figure}

\begin{itemize}
\item  \textbf{Publication de services}
\end{itemize}

Le diagramme de séquence suivant illustre les interactions nécessaires pour la
publication d'un  service. Après l'authentification, le fournisseur peut déposer
un service tout en remplissant un formulaire de dépôt d'un service. Si le
fournisseur oublie de remplir l'un des champs ou remplit un champ dans une case
inapproprié le service ne sera pas déposé.

\begin{figure}[H]
    \centering
    \includegraphics[width=1\textwidth]{images/sequence diag publier service.jpg}
    \caption{Diagramme de séquence du cas d'utilisation "Publication de service"}
    \label{fig:my_label}
\end{figure}

\begin{itemize}
    \item \textbf{Recherche de services}
\end{itemize}

Le diagramme de séquence suivant illustre les interactions nécessaires pour la
recherche d'un service. En effet, l'utilisateur saisit la recherche, le système
vérifiera si le service cherché existe dans la base de données ou pas, et il
lui affiche un résultat.

\begin{figure}[H]
    \centering
    \includegraphics[width=1\textwidth]{images/sequence diag recherche service.jpg}
    \caption{Diagramme de séquence du cas d'utilisation "Recherche de service"}
    \label{fig:my_label}
\end{figure}

\begin{itemize}
    \item \textbf{Suppression de service}
\end{itemize}

Le diagramme de séquence suivant illustre les interactions nécessaires pour la
suppression d'un service. En effet le fournisseur choisit un service, le système
demande à le fournisseur  une confirmation de suppression, si confirmée, le
système vérifie l'existence de service dans la base de données, si le service
existe la suppression va être appliquée sinon un message d'erreur sera
affiché.

\begin{figure}[H]
    \centering
    \includegraphics[width=1\textwidth]{images/sequence diag supp service.jpg}
    \caption{Diagramme de séquence du cas d'utilisation "Suppression de service"}
    \label{fig:my_label}
\end{figure}

\begin{itemize}
    \item \textbf{Modification de compte}
\end{itemize}

Le diagramme de séquence suivant illustre les interactions nécessaires pour
qu'un fournisseur modifie les informations de son compte. En effet, le
fournisseur demande au système la modification de son compte, ensuite le
système affiche le formulaire de modification et le fournisseur modifie ses
informations, le système vérifie à nouveau les informations, si elles sont
correctes, la base de données sera mise à jour sinon un message d'erreur sera
affiché.

\begin{figure}[H]
    \centering
    \includegraphics[width=1\textwidth]{images/sequence diag modifier compte.jpg}
    \caption{Diagramme de séquence du cas d'utilisation "Modification de compte"}
    \label{fig:my_label}
\end{figure}

\begin{itemize}
    \item \textbf{Suppression d'un utilisateur}
\end{itemize}

Le diagramme de séquence suivant illustre les interactions nécessaires pour la
suppression d'un utilisateur. En effet, l'administrateur choisit l'utilisateur à
supprimer, le système vérifie son existence dans la base de données, si
l'utilisateur existe la suppression sera appliquée sinon un message d'erreur
sera affiché.

\begin{figure}[H]
    \centering
    \includegraphics[width=1\textwidth]{images/sequence diag supp user.jpg}
    \caption{Diagramme de séquence du cas d'utilisation "Suppression d'un utilisateur"}
    \label{fig:my_label}
\end{figure}

\subsection{Diagramme de classe}

Le digramme de classe permet de donner la représentation statique du système à
développer. Cette représentation est centrée sur les concepts de classes et
d'associations. Chaque classe est décrite par les données et les traitements dont
elle est responsable. Nous avons pu construire le diagramme de classe indiqué
ci-dessous ensuite le passage au modèle logique est effectué pour former la base
de données de l'application.\rmq{Après avoir construit le diagramme de classe
sur la Figure~\ref{fig:diag_classes}, nous avons établi un modèle relationnel
que nous avons utilisé pour créer les différentes tables de notre base de
données.}

  \begin{figure}[H]
    \centering
    \includegraphics[width=1\textwidth]{images/Diag de classe.jpg}
    \caption{Diagramme de classe"}
    \label{fig:diag_classes}
\end{figure}

\subsection{Le schéma relationnel}

Pour réaliser ce passage, nous avons suivi des règles strictes et précises
permettant de traduire le contenu conceptuel du diagramme de classe en modèle
relationnel. Ces règles sont:

\subsubsection{Règle de passage du diagramme de classe au modèle relationnel}

Nous avons pu passer du diagramme de classes au modèle relationnel en nous
basant sur les règles suivantes:

\begin{description}
    \item[Règle 1 - Transformation des classes :] Chaque classe devient une table.
        L'identifiant de la classe devient la clé primaire et chaque attribut de
        classe se transforme en un champ de table.
    
    \item[Règle 2 - Association plusieurs-à-plusieurs:] Toute association sous
        forme ( ?..*) à ( ?..*), donne naissance à trois tables, dont la table
        dérivée de l'entité association possédera les clés primaires des deux
        autres tables comme clé primaire, plus ses propres attributs s'il s'agit
        d'une classe d'association.

    \item[Règle 3 - Association d'un-à-plusieurs:] Toute association binaire (
        ?..1) à ( ?..*) donne naissance à une table dérivée de l'entité de la
        cardinalité (1..1), et sa clé primaire est déposé comme clé étrangère
        dans la table dérivée de l'entité de la cardinalité ( ?..*).
\end{description}

\subsubsection{Le modèle logique de données}

\rmq{POUR\_QUOI UNE IMAGE POUR DU TEXTE?!}

\begin{figure}[H]
\centering
\includegraphics[width=1\textwidth]{images/MLD.png}
\caption{Modèle logique de données}
\label{fig:my_label}
\end{figure}

\subsection{Conclusion}
Dans ce chapitre, nous avons présenté le processus de développement UP, le
langage de modélisation UML, et avons procédé à la modélisation des différents
diagrammes: diagrammes de cas d'utilisations, diagrammes de séquences, et
diagrammes de classes. Cette étape de modélisation nous a permis d'avoir une vue
générale sur le comportement théorique des fonctionnalités offertes par notre
application.  Cette base théorique nous servira de guide pour le développement
de l'application, qui fera l'objet du chapitre suivant.

%Fin du chapitre 2

%Début du chapitre 3

\chapter{Réalisation et implémentions}

\section{Introduction}

Depuis le début de ce projet, nous avons bien déterminé les perspectives de
l'application en commençant par une étude préalable, qui nous a permis d'avoir
un objectif bien concret. Puis en arrivant à l'analyse, nous avons pu avoir une
idée bien claire sur comment va être notre application web, et c'est ce qui nous
a mené à mieux comprendre ses fonctionnalités. Dans ce chapitre, nous allons
nous appuyer sur la conception réalisée dans le chapitre précédent afin de
présenter la concrétisation de l'application à travers son implémentions. Nous
allons commencer par présenter les différents langages de développement ainsi
que les outils de travail utilisés dans notre démarche, nous allons ensuite
présenter en détail le principe de fonctionnement de ses outils, nous allons
finir en présentant les interfaces des pages web de notre application ainsi que
les différentes actions possibles à effectuer dans ces pages.
    
\section{Outils et environnement de développement}

\subsection{XAMPP}

XAMPP est l'environnement de développement PHP le plus populaire (\rmq{sous
windows}). XAMPP est une distribution Apache entièrement gratuite et facile à
installer contenant MySQL, PHP et Perl. Le paquetage open source XAMPP a été mis
au point pour être incroyablement facile à installer et à utiliser.

\begin{figure}[H]
    \centering
    \includegraphics[width=0.5\textwidth]{images/1024px-Xampp_logo.svg.png}
    \caption{Logo de XAMPP}
    \label{fig:my_label}
\end{figure}
        
\subsection{MYSQL}
        
MySQL est un système de gestion de base de données relationnel, un langage de
requêtes vers les bases de données exploitant le modèle relationnel et utilise
le langage SQL comme langage de requête. SQL est un langage de manipulation de
bases de données mis au point dans la années 70 par IBM, il permet d'exécuter
trois types de manipulations :

\begin{enumerate}
    \item La manipulation des tables : Création, suppression, modification de la
        structure.
    \item Les manipulations des données de la base : Sélection, modification,
        suppression d'enregistrement.
    \item La gestion des droits d'accès aux tables : Contrôle des données, droit
        d'accès, validation des modifications. \cite{memoire}
\end{enumerate}

\begin{figure}[H]
    \centering
    \includegraphics[width=0.5\textwidth]{images/1200px-MySQL.svg.png}
    \caption{Logo de MySQL}
    \label{fig:my_label}
\end{figure}

\subsection{IDE Visual Studio Code}

Visual Studio Code est un éditeur de code extensible développé par Microsoft
pour Windows, Linux et macOS. Les fonctionnalités incluent la prise en charge du
débogage, la mise en évidence de la syntaxe, la complétion intelligente du code,
les snippets, la refactorisation du code et Git intégré. \cite{ide}

\begin{figure}[H]
    \centering
    \includegraphics[width=0.3\textwidth]{images/1024px-Visual_Studio_Code_1.35_icon.svg.png}
    \caption{Logo de VS Code}
    \label{fig:my_label}
\end{figure}

\subsection{Draw.io}

C'est une application gratuite en ligne, accessible via le navigateur qui permet
de dessiner des diagrammes ou des organigrammes. Cet outil nous propose de
concevoir toutes sortes de diagrammes, de dessins vectoriels, de les enregistrer
ou de les exporter en format image, XML,PDF, etc.

\begin{figure}[H]
    \centering
    \includegraphics[width=0.5\textwidth]{images/drawio.jpg}
    \caption{Logo de Draw.io}
    \label{fig:my_label}
\end{figure}

\subsection{Git}

Git est un logiciel de gestion de versions décentralisé. C'est un logiciel libre
créé par Linus Torvalds, auteur du noyau Linux, et distribué selon les termes de
la licence publique générale GNU version 2. Le principal contributeur actuel de
git et depuis plus de 16 ans est Junio C Hamano. En 2016, il s'agit du logiciel
de gestion de versions le plus populaire qui est utilisé par plus de douze
millions de personnes.

\begin{figure}[H]
    \centering
    \includegraphics[width=0.5\textwidth]{images/1280px-Git-logo.svg.png}
    \caption{Logo de Git}
    \label{fig:my_label}
\end{figure}

\subsection{Langages de programmation}

\subsubsection{HTML}

HTML signifie « HyperText Markup Language » qu'on peut traduire par « langage de
balises pour l'hypertexte ». Il est utilisé afin de créer et de représenter le
contenu d'une page web et sa structure. D'autres technologies sont utilisées
avec HTML pour décrire la présentation d'une page (CSS) et/ou ses
fonctionnalités interactives (JavaScript).\cite{html}

HTML fonctionne grâce à des « balises » qui sont insérées au sein d'un texte
normal. Chacune de ces balises indique la signification de telle ou telle
portion de texte dans le site, par exemple <head>, <title>, <body>, <header>,
<footer>, <article>, <section>, <p>, <div>, <span>, <img> et bien d'autres
encore. Ces éléments forment les blocs utilisés pour construire un site
web.\cite{html}

On parle d'« hypertexte » en référence aux liens qui connectent les pages web
entre elles. \cite{html}

\subsubsection{CSS }

Cascading Style Sheets (CSS) est un langage de feuille de style utilisé pour
décrire la présentation d'un document écrit en HTML ou en XML. CSS décrit la
façon dont les éléments doivent être affichés à l'écran, sur du papier, en
vocalisation, ou sur d'autres supports.\cite{css}

CSS est l'un des langages principaux du Web ouvert et a été standardisé par le
W3C. Ce standard évolue sous forme de niveaux (levels), CSS1 est désormais
considéré comme obsolète, CSS2.1 correspond à la recommandation et CSS3, qui est
découpé en modules plus petits, est en voie de standardisation.\cite{css}

\subsubsection{JavaScript}

JavaScript (qui est souvent abrégé en « JS ») est un langage de script léger,
orienté objet, principalement connu comme le langage de script des pages web.
Mais il est aussi utilisé dans de nombreux environnements extérieurs aux
navigateurs web tels que Node.js, Apache CouchDB voire Adobe Acrobat. \cite{js}

Il est développé par Netscape et utilisé dans des millions de pages web et
d'applications serveur dans le monde entier.\cite{js}

Contrairement à une conception populaire, JavaScript n'est pas « du Java
interprété ». En quelques mots, JavaScript est un langage de script dynamique
utilisant une construction d'objets basée sur des prototypes.\cite{js}

\subsubsection{PHP }

PHP (Hypertext Preprocessor) est un langage de script Web intégré au HTML. Cela
signifie que le code PHP peut être inséré dans le code HTML d'une page Web.
\cite{php}

Lors de l' accès à une page PHP , le code PHP est lu ou "analysé" par le serveur
sur lequel la page réside. Les résultats des fonctions PHP de la page sont
généralement renvoyés sous forme de code HTML, qui peut être lu par le
navigateur.\cite{php}

Étant donné que le code PHP est transformé en HTML avant le chargement de la
page, les utilisateurs ne peuvent pas afficher le code PHP sur une page. Cela
rend les pages PHP suffisamment sécurisées pour accéder aux bases de données et
autres informations sécurisées.\cite{php}

\subsubsection{Framworks et Bibliothèques utilisé}

Un Framework est, comme son nom l'indique en anglais, un «  cadre de travail » .
L'objectif d'un Framework est généralement de simplifier le travail des
développeurs informatiques, en leur offrant une architecture "prête à l'emploi"
et qui leur permettre de ne pas repartir de zéro à chaque nouveau projet.

Les Frameworks sont comparables aux patrons de couture. Les principaux avantages
sont donc :

\begin{itemize}
    \item La réutilisation des codes.
    \item La standardisation de la programmation.
    \item La formalisation d'une architecture adaptée aux besoins de chaque
        entreprise.
\end{itemize}

À noter aussi que les Frameworks sont toujours « enrichis » de l'expérience de
tous les développements antérieurs.

Nous avons utilisé deux Frameworks :

\subsubsection{Codeigniter 4}

CodeIgniter est un framework de développement d'applications - une boîte à
outils - pour les personnes qui créent des sites Web en utilisant PHP. Son
objectif est de vous permettre de développer des projets beaucoup plus
rapidement que si vous écriviez du code à partir de zéro, en fournissant un
riche ensemble de bibliothèques pour les tâches courantes, ainsi qu'une
interface simple et une structure logique pour accéder à ces bibliothèques.
CodeIgniter vous permet de vous concentrer de manière créative sur votre projet
en minimisant la quantité de code nécessaire pour une tâche donnée.

Dans la mesure du possible, CodeIgniter est resté aussi flexible que possible,
vous permettant de travailler comme vous le souhaitez, sans être obligé de
travailler d'une certaine manière. Le cadre peut avoir des pièces de base
facilement étendues ou complètement remplacées pour que le système fonctionne
comme vous en avez besoin. En bref, CodeIgniter est le framework malléable qui
essaie de fournir les outils dont vous avez besoin tout en restant à l'écart.

\begin{figure}[H]
    \centering
    \includegraphics[width=1\textwidth]{images/ci-new-logo-04-02.jpg}
    \caption{Logo de CodeIgniter}
    \label{fig:my_label}
\end{figure}

CodeIgniter  requière l'installation d'un gestionnaire de dépendances « composer
».

Pour installer CodeIgniter  nous devons écrire dans le composer la commande
suivante :

\begin{figure}[H]
    \centering
    \includegraphics[width=1\textwidth]{images/codeigniter.png}
    \label{fig:my_label}
\end{figure}

\begin{itemize}
    \item \textbf{Organisation de Codeigniter}
    CodeIgniter  est composé d'une hiérarchie de dossiers et fichiers
        représentée dans cette figure :
    
    \begin{figure}[H]
        \centering
        \includegraphics[width=0.7\textwidth]{images/ci structure.png}
        \caption{Structure de Codeigniter}
        \label{fig:my_label}
    \end{figure}

    \item \textbf{Les Controllers}
    
        La grande majorité des actions qu'un utilisateur enclenche mènera à
        l'appel d'une méthode d'un contrôleur. Les  Controllers   vont permettre
        de gérer les différentes actions demandées par les utilisateurs lors
        d'une action
    
    \begin{figure}[H]
        \centering
        \includegraphics[width=0.7\textwidth]{images/controlers.png}
        \caption{Les controllers}
        \label{fig:my_label}
    \end{figure}
    
    \item \textbf{Les Models}
    \begin{itemize}
        \item  Le Model est la partie qui permet de stocker et de traiter les
            donnes.  D'une manière générale en PHP et avec CodeIgniter, le model
            n'effectue que les opérations en base de données.
        \item Pour créer un Modele avec CodeIgniter, il faut créer un fichier
            PHP dans le dossier system/application/models contenant une classe
            qui hérite de la classe Model du framework de CodeIgniter. Dans
            cette classe, on crée les fonctions qui effectuent les traitements
            sur les donnes et qui appellent les fonctions se chargeant de
            stocker les donnes (insertion SQL, écriture dans un fichier texte ou
            XML) ainsi qu'un constructeur appelant le constructeur de la classe
            Model.
    \end{itemize}

      \begin{figure}[H]
        \centering
        \includegraphics[width=0.7\textwidth]{images/models.png}
        \caption{Les Models}
        \label{fig:my_label}
    \end{figure}
    
    \item \textbf{Les Vues}
    
    Pour afficher correctement une page Web dans le navigateur, il faut utiliser
        une vue. C'est la vue qui est en charge de générer le code HTML.  Pour
        créer une vue avec CodeIgniter, il suffit de créer un script PHP ou
        HTML, qui affiche les donnes souhaites, dans le dossier
        system/application/views. Il n'y a pas de conventions particulières.
 
    \begin{figure}[H]
        \centering
        \includegraphics[width=0.7\textwidth]{images/viewq.png}
        \caption{Les Vues}
        \label{fig:my_label}
    \end{figure}
\end{itemize}
    

\subsubsection{TailwindCss}

Tailwind CSS se décrit lui-même comme un premier framework CSS utilitaire.
Plutôt que de se concentrer sur la fonctionnalité de l'élément stylisé, Tailwind
est centré sur la façon dont il doit être affiché. Cela permet au développeur de
tester plus facilement de nouveaux styles et de modifier la mise en page. Nous
interrogeons Jen Looper sur Tailwind et sur la façon dont cette approche peut
nous aider !

\begin{figure}[H]
    \centering
    \includegraphics[width=1\textwidth]{images/tailwind-css-logo-vector.png}
    \caption{Logo de Tailwindcss}
    \label{fig:my_label}
\end{figure}

\subsubsection{La bibliothéque Grocery CRUD}

Grocery CRUD est une bibliothéque Open Source qui permet à l'aide du framework
Codeigniter de crée un panel administrateur facilement avec quelques lignes de
code.

\begin{figure}[H]
    \centering
    \includegraphics[width=1\textwidth]{images/logo-big.png}
    \caption{Logo de GroceryCRUD}
    \label{fig:my_label}
\end{figure}

\begin{itemize}
    \item \textbf{Example :}
    Un exemple très simple pour commencer est les 4 lignes de code ci-dessous.
    Voici les fonctions les plus courantes :
    \begin{figure}[H]
    \centering
    \includegraphics[width=1\textwidth]{images/example.png}
    \caption{Exemple d'utilisation de GroceryCrud}
    \label{fig:my_label}
\end{figure}
\end{itemize}

\section{Captures d'écran et fonctionnement}

\subsection{La page d'accueil}
    
La première interface que l'utilisateur rencontre lors de l'accès vers notre
application est celle de l'accueil. L'utilisateur aura deux choix, soit de se
connecter en cliquant sur le bouton "Connexion" s'il à déjà un compte, soit de
s'inscrire en cliquant sur le bouton "Inscription" si l'utilisateur n'a pas de
compte.

Si l'utilisateur choisi de se connecter, il sera rediriger vers l'interface
d'authentification, ou l'utilisateur sera amener a saisir ses information, dans
le cas ou ses informations sont correct, la page principale sera afficher, dans
le cas contraire, un message d'erreur sera afficher.

L'utilisateur en fonction de sont type (Admin, Fournisseur, Client) sera
automatiquement rediriger vers la page principale qui lui convient

\begin{figure}[H]
    \centering
    \includegraphics[width=1\textwidth]{images/accueil.png}
    \caption{Page d'accueil}
\end{figure}

\subsection{La page d'inscription}

Si l'utilisateur choisi de se s'inscrire, il sera rediriger vers l'interface
d'inscription, ou l'utilisateur sera amener remplir un formulaire d'inscription
en mentionnant son type au préalable, le système vas vérifier la validité des
informations saisie et les insèrent dans la base de donnée.

\begin{figure}[H]
    \centering
    \includegraphics[width=1\textwidth]{images/inscription.png}
    \caption{Page de pré-inscription}
\end{figure}

\begin{figure}[H]
    \centering
    \includegraphics[width=1\textwidth]{images/formulaire.png}
    \caption{Formulaire d'inscription}
\end{figure}

\subsection{Interfaces coté client}

L'interface principale du client se présente comme suit :
\begin{figure}[H]
    \centering
    \includegraphics[width=1\textwidth]{images/client dash.png}
    \caption{Interface principale du client}
\end{figure}

Depuis cette interface, le client pourras :
\begin{itemize}
    \item Chercher des services selon un filtre
    \item Voir les messages échanger
    \item Voir les commandes effectuer et leurs état
    \item voir les avis précédemment laisser.
\end{itemize}

\begin{figure}[H]
    \centering
    \includegraphics[width=1\textwidth]{images/messages.png}
    \caption{Interface des messages échanger}
\end{figure}

\begin{figure}[H]
    \centering
    \includegraphics[width=1\textwidth]{images/services.png} 
    \caption{Interface des services}
\end{figure}

Le client pourras consulter le contenu du service (Descriptions, images, tarifs)
et pourras commander le service, contacter le fournisseur et laisser un avis.

 \begin{figure}[H]
    \centering
    \includegraphics[width=1\textwidth]{images/service consult.png} 
    \caption{Interface des services}
\end{figure}

\subsection{Interface coté fournisseur}

L'interface principale du fournisseur se présente comme suit :

\begin{figure}[H]
    \centering
    \includegraphics[width=1\textwidth]{images/fourni main.png} 
    \caption{Interface coté Fournisseur}
\end{figure}

Le fournisseur pourras a travers cette interface : 
\begin{itemize}
    \item Voir l'ensemble de ses services et ajouter au besoin
    \item Gérer les commandes reçu
    \item Échanger des messages
    \item Voir quelques statistiques
    
\end{itemize}

\subsection{Conclusion}

Après le chapitre de l'analyse et de la conception, vient la réalisation de
l'application. Dans cette partie technique, nous avons émit les outils et les
environnements utilisé lors de la réalisation, ainsi que les différents langages
de programmation. Puis nous avons citer quelques fonctionnalités de notre
application en s'appuyant sur des capture d'écrans et quelque explication pour
clarifier le fonctionnement des ses fonctionnalités

%Fin du chapitre 3

\section{Conclusion Générale}

Nous sommes parvenus, grâce à ce projet à réaliser une application web sur le
thème ‘ Conception et réalisation d'une application web d'offres de services
professionnel ‘  destinée a tout le monde qui souhait chercher quelqu'un de
qualifier, de professionnel.

Afin d'y parvenir, nous avons débuté notre projet par évoqué la problématique
rencontrer par la plus part des gens lors de leurs recherche de personnes
qualifié offrant leurs services, donner les solutions proposées par notre
application, en précisant ses différentes fonctionnalité, dans le but de mieux
comprendre notre travail. 

Dans le deuxième chapitre, nous évoquons des définitions de l'UML et UP ensuite
nous avons déterminer les différents diagrammes d'UML afin de représenter la
structure globale et l'architecture détaillé de notre système en détaillant son
fonctionnement.

En finissant par un troisième chapitre, où nous avons présenté les différents
outils et langages de développement utilisés pour la réalisation de
l'application, ainsi qu'une présentation de quelques interfaces de celle ci.
    
Dans notre application, nous avons essayé de régler les différents problèmes
que rencontrent les gens au quotidien; (\rmq{non, ce n'est pas vrai.})

\begin{itemize}
    \item Manque d'informations préalables sur une personne.
    \item Manque de moyens pour jauger les compétences d'une personne.
    \item Manque de moyens de contacter directement une personne.
\end{itemize}

\printbibliography

\end{document}
